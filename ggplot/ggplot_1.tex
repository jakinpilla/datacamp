\documentclass[]{article}
\usepackage{lmodern}
\usepackage{amssymb,amsmath}
\usepackage{ifxetex,ifluatex}
\usepackage{fixltx2e} % provides \textsubscript
\ifnum 0\ifxetex 1\fi\ifluatex 1\fi=0 % if pdftex
  \usepackage[T1]{fontenc}
  \usepackage[utf8]{inputenc}
\else % if luatex or xelatex
  \ifxetex
    \usepackage{mathspec}
  \else
    \usepackage{fontspec}
  \fi
  \defaultfontfeatures{Ligatures=TeX,Scale=MatchLowercase}
\fi
% use upquote if available, for straight quotes in verbatim environments
\IfFileExists{upquote.sty}{\usepackage{upquote}}{}
% use microtype if available
\IfFileExists{microtype.sty}{%
\usepackage{microtype}
\UseMicrotypeSet[protrusion]{basicmath} % disable protrusion for tt fonts
}{}
\usepackage[margin=1in]{geometry}
\usepackage{hyperref}
\hypersetup{unicode=true,
            pdftitle={ggplot\_1},
            pdfauthor={Daniel\_Kim},
            pdfborder={0 0 0},
            breaklinks=true}
\urlstyle{same}  % don't use monospace font for urls
\usepackage{color}
\usepackage{fancyvrb}
\newcommand{\VerbBar}{|}
\newcommand{\VERB}{\Verb[commandchars=\\\{\}]}
\DefineVerbatimEnvironment{Highlighting}{Verbatim}{commandchars=\\\{\}}
% Add ',fontsize=\small' for more characters per line
\usepackage{framed}
\definecolor{shadecolor}{RGB}{248,248,248}
\newenvironment{Shaded}{\begin{snugshade}}{\end{snugshade}}
\newcommand{\AlertTok}[1]{\textcolor[rgb]{0.94,0.16,0.16}{#1}}
\newcommand{\AnnotationTok}[1]{\textcolor[rgb]{0.56,0.35,0.01}{\textbf{\textit{#1}}}}
\newcommand{\AttributeTok}[1]{\textcolor[rgb]{0.77,0.63,0.00}{#1}}
\newcommand{\BaseNTok}[1]{\textcolor[rgb]{0.00,0.00,0.81}{#1}}
\newcommand{\BuiltInTok}[1]{#1}
\newcommand{\CharTok}[1]{\textcolor[rgb]{0.31,0.60,0.02}{#1}}
\newcommand{\CommentTok}[1]{\textcolor[rgb]{0.56,0.35,0.01}{\textit{#1}}}
\newcommand{\CommentVarTok}[1]{\textcolor[rgb]{0.56,0.35,0.01}{\textbf{\textit{#1}}}}
\newcommand{\ConstantTok}[1]{\textcolor[rgb]{0.00,0.00,0.00}{#1}}
\newcommand{\ControlFlowTok}[1]{\textcolor[rgb]{0.13,0.29,0.53}{\textbf{#1}}}
\newcommand{\DataTypeTok}[1]{\textcolor[rgb]{0.13,0.29,0.53}{#1}}
\newcommand{\DecValTok}[1]{\textcolor[rgb]{0.00,0.00,0.81}{#1}}
\newcommand{\DocumentationTok}[1]{\textcolor[rgb]{0.56,0.35,0.01}{\textbf{\textit{#1}}}}
\newcommand{\ErrorTok}[1]{\textcolor[rgb]{0.64,0.00,0.00}{\textbf{#1}}}
\newcommand{\ExtensionTok}[1]{#1}
\newcommand{\FloatTok}[1]{\textcolor[rgb]{0.00,0.00,0.81}{#1}}
\newcommand{\FunctionTok}[1]{\textcolor[rgb]{0.00,0.00,0.00}{#1}}
\newcommand{\ImportTok}[1]{#1}
\newcommand{\InformationTok}[1]{\textcolor[rgb]{0.56,0.35,0.01}{\textbf{\textit{#1}}}}
\newcommand{\KeywordTok}[1]{\textcolor[rgb]{0.13,0.29,0.53}{\textbf{#1}}}
\newcommand{\NormalTok}[1]{#1}
\newcommand{\OperatorTok}[1]{\textcolor[rgb]{0.81,0.36,0.00}{\textbf{#1}}}
\newcommand{\OtherTok}[1]{\textcolor[rgb]{0.56,0.35,0.01}{#1}}
\newcommand{\PreprocessorTok}[1]{\textcolor[rgb]{0.56,0.35,0.01}{\textit{#1}}}
\newcommand{\RegionMarkerTok}[1]{#1}
\newcommand{\SpecialCharTok}[1]{\textcolor[rgb]{0.00,0.00,0.00}{#1}}
\newcommand{\SpecialStringTok}[1]{\textcolor[rgb]{0.31,0.60,0.02}{#1}}
\newcommand{\StringTok}[1]{\textcolor[rgb]{0.31,0.60,0.02}{#1}}
\newcommand{\VariableTok}[1]{\textcolor[rgb]{0.00,0.00,0.00}{#1}}
\newcommand{\VerbatimStringTok}[1]{\textcolor[rgb]{0.31,0.60,0.02}{#1}}
\newcommand{\WarningTok}[1]{\textcolor[rgb]{0.56,0.35,0.01}{\textbf{\textit{#1}}}}
\usepackage{graphicx,grffile}
\makeatletter
\def\maxwidth{\ifdim\Gin@nat@width>\linewidth\linewidth\else\Gin@nat@width\fi}
\def\maxheight{\ifdim\Gin@nat@height>\textheight\textheight\else\Gin@nat@height\fi}
\makeatother
% Scale images if necessary, so that they will not overflow the page
% margins by default, and it is still possible to overwrite the defaults
% using explicit options in \includegraphics[width, height, ...]{}
\setkeys{Gin}{width=\maxwidth,height=\maxheight,keepaspectratio}
\IfFileExists{parskip.sty}{%
\usepackage{parskip}
}{% else
\setlength{\parindent}{0pt}
\setlength{\parskip}{6pt plus 2pt minus 1pt}
}
\setlength{\emergencystretch}{3em}  % prevent overfull lines
\providecommand{\tightlist}{%
  \setlength{\itemsep}{0pt}\setlength{\parskip}{0pt}}
\setcounter{secnumdepth}{0}
% Redefines (sub)paragraphs to behave more like sections
\ifx\paragraph\undefined\else
\let\oldparagraph\paragraph
\renewcommand{\paragraph}[1]{\oldparagraph{#1}\mbox{}}
\fi
\ifx\subparagraph\undefined\else
\let\oldsubparagraph\subparagraph
\renewcommand{\subparagraph}[1]{\oldsubparagraph{#1}\mbox{}}
\fi

%%% Use protect on footnotes to avoid problems with footnotes in titles
\let\rmarkdownfootnote\footnote%
\def\footnote{\protect\rmarkdownfootnote}

%%% Change title format to be more compact
\usepackage{titling}

% Create subtitle command for use in maketitle
\providecommand{\subtitle}[1]{
  \posttitle{
    \begin{center}\large#1\end{center}
    }
}

\setlength{\droptitle}{-2em}

  \title{ggplot\_1}
    \pretitle{\vspace{\droptitle}\centering\huge}
  \posttitle{\par}
    \author{Daniel\_Kim}
    \preauthor{\centering\large\emph}
  \postauthor{\par}
      \predate{\centering\large\emph}
  \postdate{\par}
    \date{2019 8 15}


\begin{document}
\maketitle

\begin{Shaded}
\begin{Highlighting}[]
\KeywordTok{summary}\NormalTok{(iris)}
\end{Highlighting}
\end{Shaded}

\begin{verbatim}
##   Sepal.Length    Sepal.Width     Petal.Length    Petal.Width   
##  Min.   :4.300   Min.   :2.000   Min.   :1.000   Min.   :0.100  
##  1st Qu.:5.100   1st Qu.:2.800   1st Qu.:1.600   1st Qu.:0.300  
##  Median :5.800   Median :3.000   Median :4.350   Median :1.300  
##  Mean   :5.843   Mean   :3.057   Mean   :3.758   Mean   :1.199  
##  3rd Qu.:6.400   3rd Qu.:3.300   3rd Qu.:5.100   3rd Qu.:1.800  
##  Max.   :7.900   Max.   :4.400   Max.   :6.900   Max.   :2.500  
##        Species  
##  setosa    :50  
##  versicolor:50  
##  virginica :50  
##                 
##                 
## 
\end{verbatim}

\begin{Shaded}
\begin{Highlighting}[]
\NormalTok{iris }\OperatorTok\StringTok{ }\KeywordTok{as_tibble}\NormalTok{() }\OperatorTok\StringTok{ }\KeywordTok{head}\NormalTok{()}
\end{Highlighting}
\end{Shaded}

\begin{verbatim}
## # A tibble: 6 x 5
##   Sepal.Length Sepal.Width Petal.Length Petal.Width Species
##          <dbl>       <dbl>        <dbl>       <dbl> <fct>  
## 1          5.1         3.5          1.4         0.2 setosa 
## 2          4.9         3            1.4         0.2 setosa 
## 3          4.7         3.2          1.3         0.2 setosa 
## 4          4.6         3.1          1.5         0.2 setosa 
## 5          5           3.6          1.4         0.2 setosa 
## 6          5.4         3.9          1.7         0.4 setosa
\end{verbatim}

\begin{Shaded}
\begin{Highlighting}[]
\ImportTok{import}\NormalTok{ numpy }\ImportTok{as}\NormalTok{ np}
\ImportTok{import}\NormalTok{ pandas }\ImportTok{as}\NormalTok{ pd}

\NormalTok{iris }\OperatorTok{=}\NormalTok{ r.iris}
\NormalTok{iris.plot()}
\end{Highlighting}
\end{Shaded}

\includegraphics{ggplot_1_files/figure-latex/unnamed-chunk-1-1.pdf}

\begin{Shaded}
\begin{Highlighting}[]
\NormalTok{iris }\OperatorTok
\StringTok{  }\KeywordTok{as_tibble}\NormalTok{() }\OperatorTok
\StringTok{  }\KeywordTok{rowid_to_column}\NormalTok{() }\OperatorTok
\StringTok{  }\KeywordTok{gather}\NormalTok{(Sepal.Length, Sepal.Width, Petal.Length, Petal.Width, }\DataTypeTok{key =} \StringTok{"Featrures"}\NormalTok{, }\DataTypeTok{value =} \StringTok{"value"}\NormalTok{) }\OperatorTok
\StringTok{  }\KeywordTok{head}\NormalTok{()}
\end{Highlighting}
\end{Shaded}

\begin{verbatim}
## # A tibble: 6 x 4
##   rowid Species Featrures    value
##   <int> <fct>   <chr>        <dbl>
## 1     1 setosa  Sepal.Length   5.1
## 2     2 setosa  Sepal.Length   4.9
## 3     3 setosa  Sepal.Length   4.7
## 4     4 setosa  Sepal.Length   4.6
## 5     5 setosa  Sepal.Length   5  
## 6     6 setosa  Sepal.Length   5.4
\end{verbatim}

\begin{Shaded}
\begin{Highlighting}[]
\NormalTok{iris }\OperatorTok
\StringTok{  }\KeywordTok{as_tibble}\NormalTok{() }\OperatorTok
\StringTok{  }\KeywordTok{rowid_to_column}\NormalTok{() }\OperatorTok
\StringTok{  }\KeywordTok{gather}\NormalTok{(Sepal.Length, Sepal.Width, Petal.Length, Petal.Width, }\DataTypeTok{key =} \StringTok{"Features"}\NormalTok{, }\DataTypeTok{value =} \StringTok{"value"}\NormalTok{) }\OperatorTok
\StringTok{  }\KeywordTok{separate}\NormalTok{(Features, }\DataTypeTok{sep =} \StringTok{"}\CharTok{\textbackslash{}\textbackslash{}}\StringTok{."}\NormalTok{, }\DataTypeTok{into =} \KeywordTok{c}\NormalTok{(}\StringTok{"Part"}\NormalTok{, }\StringTok{"Measure"}\NormalTok{)) }\OperatorTok
\StringTok{  }\KeywordTok{spread}\NormalTok{(Measure, value) }\OperatorTok
\StringTok{  }\KeywordTok{select}\NormalTok{(}\OperatorTok{-}\NormalTok{rowid) ->}\StringTok{ }\NormalTok{iris_wide}

\NormalTok{iris_wide }\OperatorTok\StringTok{ }\KeywordTok{head}\NormalTok{()}
\end{Highlighting}
\end{Shaded}

\begin{verbatim}
## # A tibble: 6 x 4
##   Species Part  Length Width
##   <fct>   <chr>  <dbl> <dbl>
## 1 setosa  Petal    1.4   0.2
## 2 setosa  Sepal    5.1   3.5
## 3 setosa  Petal    1.4   0.2
## 4 setosa  Sepal    4.9   3  
## 5 setosa  Petal    1.3   0.2
## 6 setosa  Sepal    4.7   3.2
\end{verbatim}

\begin{Shaded}
\begin{Highlighting}[]
\NormalTok{iris}\OperatorTok{$}\NormalTok{Flower <-}\StringTok{ }\DecValTok{1}\OperatorTok{:}\KeywordTok{nrow}\NormalTok{(iris)}

\NormalTok{iris }\OperatorTok
\StringTok{  }\KeywordTok{gather}\NormalTok{(key, value, }\OperatorTok{-}\NormalTok{Species, }\OperatorTok{-}\NormalTok{Flower) }\OperatorTok
\StringTok{  }\KeywordTok{separate}\NormalTok{(key, }\KeywordTok{c}\NormalTok{(}\StringTok{"Part"}\NormalTok{, }\StringTok{"Measure"}\NormalTok{), }\StringTok{"}\CharTok{\textbackslash{}\textbackslash{}}\StringTok{."}\NormalTok{) }\OperatorTok
\StringTok{  }\KeywordTok{spread}\NormalTok{(Measure, value) ->}\StringTok{ }\NormalTok{iris.wide}

\NormalTok{iris.wide }\OperatorTok\StringTok{ }\KeywordTok{head}\NormalTok{()}
\end{Highlighting}
\end{Shaded}

\begin{verbatim}
##   Species Flower  Part Length Width
## 1  setosa      1 Petal    1.4   0.2
## 2  setosa      1 Sepal    5.1   3.5
## 3  setosa      2 Petal    1.4   0.2
## 4  setosa      2 Sepal    4.9   3.0
## 5  setosa      3 Petal    1.3   0.2
## 6  setosa      3 Sepal    4.7   3.2
\end{verbatim}

\begin{Shaded}
\begin{Highlighting}[]
\NormalTok{iris_wide }\OperatorTok{=}\NormalTok{ r.iris_wide}
\NormalTok{iris_wide[}\StringTok{'Part'}\NormalTok{].value_counts()}
\end{Highlighting}
\end{Shaded}

\begin{verbatim}
## Sepal    150
## Petal    150
## Name: Part, dtype: int64
\end{verbatim}

\begin{Shaded}
\begin{Highlighting}[]
\NormalTok{iris_wide.head()}
\end{Highlighting}
\end{Shaded}

\begin{verbatim}
##   Species   Part  Length  Width
## 0  setosa  Petal     1.4    0.2
## 1  setosa  Sepal     5.1    3.5
## 2  setosa  Petal     1.4    0.2
## 3  setosa  Sepal     4.9    3.0
## 4  setosa  Petal     1.3    0.2
\end{verbatim}

\begin{Shaded}
\begin{Highlighting}[]
\ImportTok{import}\NormalTok{ matplotlib.pyplot }\ImportTok{as}\NormalTok{ plt}
\NormalTok{plt.scatter(iris_wide[}\StringTok{'Length'}\NormalTok{], iris_wide[}\StringTok{'Width'}\NormalTok{], alpha}\OperatorTok{=}\NormalTok{.}\DecValTok{5}\NormalTok{)}
\end{Highlighting}
\end{Shaded}

\includegraphics{ggplot_1_files/figure-latex/unnamed-chunk-7-1.pdf}

\begin{Shaded}
\begin{Highlighting}[]
\NormalTok{iris }\OperatorTok
\StringTok{  }\KeywordTok{rowid_to_column}\NormalTok{() }\OperatorTok
\StringTok{  }\KeywordTok{gather}\NormalTok{(Sepal.Length, Sepal.Width, Petal.Length, Petal.Width, }\DataTypeTok{key =} \StringTok{"Features"}\NormalTok{, }\DataTypeTok{value =} \StringTok{"value"}\NormalTok{) }\OperatorTok
\StringTok{  }\KeywordTok{separate}\NormalTok{(Features, }\DataTypeTok{sep =} \StringTok{"}\CharTok{\textbackslash{}\textbackslash{}}\StringTok{."}\NormalTok{, }\DataTypeTok{into =} \KeywordTok{c}\NormalTok{(}\StringTok{"Part"}\NormalTok{, }\StringTok{"Measure"}\NormalTok{)) }\OperatorTok
\StringTok{  }\KeywordTok{spread}\NormalTok{(Species, value) ->}\StringTok{ }\NormalTok{iris_tmp}

\NormalTok{iris_tmp }\OperatorTok\StringTok{ }\KeywordTok{filter}\NormalTok{(}\OperatorTok{!}\KeywordTok{is.na}\NormalTok{(setosa)) }\OperatorTok\StringTok{ }\KeywordTok{select}\NormalTok{(}\OperatorTok{-}\NormalTok{versicolor, }\OperatorTok{-}\NormalTok{virginica) ->}\StringTok{ }\NormalTok{iris_tmp_}\DecValTok{1}
\NormalTok{iris_tmp }\OperatorTok\StringTok{ }\KeywordTok{filter}\NormalTok{(}\OperatorTok{!}\KeywordTok{is.na}\NormalTok{(versicolor)) }\OperatorTok\StringTok{ }\KeywordTok{select}\NormalTok{(versicolor) ->}\StringTok{ }\NormalTok{iris_tmp_}\DecValTok{2}
\NormalTok{iris_tmp }\OperatorTok\StringTok{ }\KeywordTok{filter}\NormalTok{(}\OperatorTok{!}\KeywordTok{is.na}\NormalTok{(virginica)) }\OperatorTok\StringTok{ }\KeywordTok{select}\NormalTok{(virginica) ->}\StringTok{ }\NormalTok{iris_tmp_}\DecValTok{3}

\NormalTok{iris_tmp_}\DecValTok{1} \OperatorTok
\StringTok{  }\KeywordTok{bind_cols}\NormalTok{(iris_tmp_}\DecValTok{2}\NormalTok{, iris_tmp_}\DecValTok{3}\NormalTok{) }\OperatorTok
\StringTok{  }\KeywordTok{select}\NormalTok{(}\OperatorTok{-}\NormalTok{rowid) ->}\StringTok{ }\NormalTok{iris_wide_}\DecValTok{2}

\NormalTok{iris_wide_}\DecValTok{2} \OperatorTok\StringTok{ }\KeywordTok{head}\NormalTok{()}
\end{Highlighting}
\end{Shaded}

\begin{verbatim}
##   Flower  Part Measure setosa versicolor virginica
## 1      1 Petal  Length    1.4        4.7       6.0
## 2      1 Petal   Width    0.2        1.4       2.5
## 3      1 Sepal  Length    5.1        7.0       6.3
## 4      1 Sepal   Width    3.5        3.2       3.3
## 5      2 Petal  Length    1.4        4.5       5.1
## 6      2 Petal   Width    0.2        1.5       1.9
\end{verbatim}

\begin{Shaded}
\begin{Highlighting}[]
\NormalTok{iris }\OperatorTok
\StringTok{  }\KeywordTok{rowid_to_column}\NormalTok{() }\OperatorTok
\StringTok{  }\KeywordTok{gather}\NormalTok{(Sepal.Length, Sepal.Width, Petal.Length, Petal.Width, }\DataTypeTok{key =} \StringTok{"Features"}\NormalTok{, }\DataTypeTok{value =} \StringTok{"value"}\NormalTok{) }\OperatorTok
\StringTok{  }\KeywordTok{separate}\NormalTok{(Features, }\DataTypeTok{sep =} \StringTok{"}\CharTok{\textbackslash{}\textbackslash{}}\StringTok{."}\NormalTok{, }\DataTypeTok{into =} \KeywordTok{c}\NormalTok{(}\StringTok{"Part"}\NormalTok{, }\StringTok{"Measure"}\NormalTok{)) }\OperatorTok
\StringTok{  }\KeywordTok{select}\NormalTok{(}\OperatorTok{-}\NormalTok{rowid) ->}\StringTok{ }\NormalTok{iris_tidy}

\NormalTok{iris_tidy }\OperatorTok\StringTok{ }\KeywordTok{head}\NormalTok{()}
\end{Highlighting}
\end{Shaded}

\begin{verbatim}
##   Species Flower  Part Measure value
## 1  setosa      1 Sepal  Length   5.1
## 2  setosa      2 Sepal  Length   4.9
## 3  setosa      3 Sepal  Length   4.7
## 4  setosa      4 Sepal  Length   4.6
## 5  setosa      5 Sepal  Length   5.0
## 6  setosa      6 Sepal  Length   5.4
\end{verbatim}

\begin{Shaded}
\begin{Highlighting}[]
\NormalTok{iris }\OperatorTok
\StringTok{  }\KeywordTok{gather}\NormalTok{(key, Value, }\OperatorTok{-}\NormalTok{Species) }\OperatorTok
\StringTok{  }\KeywordTok{separate}\NormalTok{(key, }\KeywordTok{c}\NormalTok{(}\StringTok{"Part"}\NormalTok{, }\StringTok{"Measure"}\NormalTok{), }\StringTok{"}\CharTok{\textbackslash{}\textbackslash{}}\StringTok{."}\NormalTok{) ->}\StringTok{ }\NormalTok{iris_tidy; iris_tidy }\OperatorTok\StringTok{ }\KeywordTok{head}\NormalTok{()}
\end{Highlighting}
\end{Shaded}

\begin{verbatim}
## Warning: Expected 2 pieces. Missing pieces filled with `NA` in 150 rows
## [601, 602, 603, 604, 605, 606, 607, 608, 609, 610, 611, 612, 613, 614, 615,
## 616, 617, 618, 619, 620, ...].
\end{verbatim}

\begin{verbatim}
##   Species  Part Measure Value
## 1  setosa Sepal  Length   5.1
## 2  setosa Sepal  Length   4.9
## 3  setosa Sepal  Length   4.7
## 4  setosa Sepal  Length   4.6
## 5  setosa Sepal  Length   5.0
## 6  setosa Sepal  Length   5.4
\end{verbatim}

\begin{Shaded}
\begin{Highlighting}[]
\KeywordTok{str}\NormalTok{(iris)}
\end{Highlighting}
\end{Shaded}

\begin{verbatim}
## 'data.frame':    150 obs. of  6 variables:
##  $ Sepal.Length: num  5.1 4.9 4.7 4.6 5 5.4 4.6 5 4.4 4.9 ...
##  $ Sepal.Width : num  3.5 3 3.2 3.1 3.6 3.9 3.4 3.4 2.9 3.1 ...
##  $ Petal.Length: num  1.4 1.4 1.3 1.5 1.4 1.7 1.4 1.5 1.4 1.5 ...
##  $ Petal.Width : num  0.2 0.2 0.2 0.2 0.2 0.4 0.3 0.2 0.2 0.1 ...
##  $ Species     : Factor w/ 3 levels "setosa","versicolor",..: 1 1 1 1 1 1 1 1 1 1 ...
##  $ Flower      : int  1 2 3 4 5 6 7 8 9 10 ...
\end{verbatim}

\begin{Shaded}
\begin{Highlighting}[]
\KeywordTok{str}\NormalTok{(iris_wide)}
\end{Highlighting}
\end{Shaded}

\begin{verbatim}
## Classes 'tbl_df', 'tbl' and 'data.frame':    300 obs. of  4 variables:
##  $ Species: Factor w/ 3 levels "setosa","versicolor",..: 1 1 1 1 1 1 1 1 1 1 ...
##  $ Part   : chr  "Petal" "Sepal" "Petal" "Sepal" ...
##  $ Length : num  1.4 5.1 1.4 4.9 1.3 4.7 1.5 4.6 1.4 5 ...
##  $ Width  : num  0.2 3.5 0.2 3 0.2 3.2 0.2 3.1 0.2 3.6 ...
\end{verbatim}

\begin{Shaded}
\begin{Highlighting}[]
\KeywordTok{str}\NormalTok{(iris_tidy)}
\end{Highlighting}
\end{Shaded}

\begin{verbatim}
## 'data.frame':    750 obs. of  4 variables:
##  $ Species: Factor w/ 3 levels "setosa","versicolor",..: 1 1 1 1 1 1 1 1 1 1 ...
##  $ Part   : chr  "Sepal" "Sepal" "Sepal" "Sepal" ...
##  $ Measure: chr  "Length" "Length" "Length" "Length" ...
##  $ Value  : num  5.1 4.9 4.7 4.6 5 5.4 4.6 5 4.4 4.9 ...
\end{verbatim}

\begin{Shaded}
\begin{Highlighting}[]
\KeywordTok{ggplot}\NormalTok{(iris_tidy, }\KeywordTok{aes}\NormalTok{(}\DataTypeTok{x =}\NormalTok{ Species, }\DataTypeTok{y =}\NormalTok{ Value, }\DataTypeTok{col =}\NormalTok{ Part)) }\OperatorTok{+}\StringTok{ }
\StringTok{  }\KeywordTok{geom_jitter}\NormalTok{() }\OperatorTok{+}\StringTok{ }
\StringTok{  }\KeywordTok{facet_grid}\NormalTok{(. }\OperatorTok{~}\StringTok{ }\NormalTok{Measure)}
\end{Highlighting}
\end{Shaded}

\includegraphics{ggplot_1_files/figure-latex/unnamed-chunk-14-1.pdf}

\begin{Shaded}
\begin{Highlighting}[]
\KeywordTok{ggplot}\NormalTok{(iris_wide, }\KeywordTok{aes}\NormalTok{(Length, Width, }\DataTypeTok{color =}\NormalTok{ Part)) }\OperatorTok{+}\StringTok{ }\KeywordTok{geom_jitter}\NormalTok{() }\OperatorTok{+}\StringTok{ }\KeywordTok{facet_grid}\NormalTok{(. }\OperatorTok{~}\StringTok{ }\NormalTok{Species)}
\end{Highlighting}
\end{Shaded}

\includegraphics{ggplot_1_files/figure-latex/unnamed-chunk-15-1.pdf}

\begin{Shaded}
\begin{Highlighting}[]
\NormalTok{mtcars }\OperatorTok\StringTok{ }\KeywordTok{head}\NormalTok{()}
\end{Highlighting}
\end{Shaded}

\begin{verbatim}
##                    mpg cyl disp  hp drat    wt  qsec vs am gear carb
## Mazda RX4         21.0   6  160 110 3.90 2.620 16.46  0  1    4    4
## Mazda RX4 Wag     21.0   6  160 110 3.90 2.875 17.02  0  1    4    4
## Datsun 710        22.8   4  108  93 3.85 2.320 18.61  1  1    4    1
## Hornet 4 Drive    21.4   6  258 110 3.08 3.215 19.44  1  0    3    1
## Hornet Sportabout 18.7   8  360 175 3.15 3.440 17.02  0  0    3    2
## Valiant           18.1   6  225 105 2.76 3.460 20.22  1  0    3    1
\end{verbatim}

\begin{Shaded}
\begin{Highlighting}[]
\KeywordTok{ggplot}\NormalTok{(mtcars, }\KeywordTok{aes}\NormalTok{(mpg, cyl)) }\OperatorTok{+}\StringTok{ }\KeywordTok{geom_point}\NormalTok{()}
\end{Highlighting}
\end{Shaded}

\includegraphics{ggplot_1_files/figure-latex/unnamed-chunk-16-1.pdf}

\begin{Shaded}
\begin{Highlighting}[]
\KeywordTok{ggplot}\NormalTok{(mtcars, }\KeywordTok{aes}\NormalTok{(cyl, mpg)) }\OperatorTok{+}\StringTok{ }\KeywordTok{geom_point}\NormalTok{()}
\end{Highlighting}
\end{Shaded}

\includegraphics{ggplot_1_files/figure-latex/unnamed-chunk-17-1.pdf}

\begin{Shaded}
\begin{Highlighting}[]
\KeywordTok{ggplot}\NormalTok{(mtcars, }\KeywordTok{aes}\NormalTok{(wt, mpg, }\DataTypeTok{col =}\NormalTok{ cyl)) }\OperatorTok{+}\StringTok{ }\KeywordTok{geom_point}\NormalTok{()}
\end{Highlighting}
\end{Shaded}

\includegraphics{ggplot_1_files/figure-latex/unnamed-chunk-18-1.pdf}

\begin{Shaded}
\begin{Highlighting}[]
\KeywordTok{ggplot}\NormalTok{(mtcars, }\KeywordTok{aes}\NormalTok{(wt, mpg, }\DataTypeTok{col =}\NormalTok{ cyl)) }\OperatorTok{+}\StringTok{ }\KeywordTok{geom_point}\NormalTok{(}\DataTypeTok{shape =} \DecValTok{1}\NormalTok{, }\DataTypeTok{size =} \DecValTok{4}\NormalTok{)}
\end{Highlighting}
\end{Shaded}

\includegraphics{ggplot_1_files/figure-latex/unnamed-chunk-19-1.pdf}


\end{document}
